\documentclass[11pt]{llncs}
\usepackage{preamble}

\begin{document}
\title{
Set Theory Class 2018 - 2019\\
Exercise Set 1}
\date{\today}
\author{Dionysis Zindros\\
    \email{dionyziz@di.uoa.gr}}
\institute{
National and Kapodistrian University of Athens}
\maketitle
\noindent
\makebox[\linewidth]{\small \today}

\thispagestyle{plain}

\section*{Exercise 1}
\begin{lemma}
$\left<A, B\right> = \left<C, D\right> \leftrightarrow (A = C \land B = D)$.
\end{lemma}
\begin{proof}
  \item $(\leftarrow)$
  Suppose $A = C \land B = D$. Then $A = C$ and $B = D$. Therefore:

  \begin{align*}
                &\left<A, B\right> = \left<A, B\right>\\
    \Rightarrow &\left<A, B\right> = \left<C, B\right>\\
    \Rightarrow &\left<A, B\right> = \left<C, D\right>
  \end{align*}

  \item $(\rightarrow)$
  Suppose $\left<A, B\right> = \left<C, D\right>$. By the Kuratowski definition:

  \begin{alignat*}{3}
                       &&\left<A, B\right> &= \left<C, D\right>\\
    & \Rightarrow\quad &\{\{A\}, \{A, B\}\} &= \{\{C\}, \{C, D\}\}
  \end{alignat*}

  We distinguish two cases.

  \textbf{Case 1:} $A = B$.

  Then $\{A, B\} = \{A, A\} = \{A\}$ and so $\{\{A\}, \{A, B\}\} = \{\{A\}\}$.
  Hence, $\{\{A\}\} = \{\{C\}, \{C, D\}\}$ and by extensionality
  $\{A\} = \{C, D\}$. Hence $C = D = A$ and we are done.

  \textbf{Case 2:} $A \neq B$.

  Then $\{A\} \neq \{A, B\}$, hence by extensionality $\{C\} \neq \{C, D\}$.
  But it cannot be that $\{A\} = \{C, D\}$ nor $\{A, B\} = \{C\}$,
  therefore $\{A\} = \{C\}$ and $\{A, B\} = \{C, D\}$. Hence $A = C$ and
  $B = D$.
  \qed
\end{proof}

\section*{Exercise 2}

\begin{lemma}
  \[
    \left(\bigcup (A_i)_{i \in I}\right)' = \bigcap\left(A_i'\right)_{i \in I}
  \]
\end{lemma}
\begin{proof}
  By extensionality and the definition of $\bigcup, \bigcap$.

  \begin{minipage}[t]{0.49\textwidth}
    \begin{alignat*}{3}
                            && x &\in \left(\bigcup (A_i)_{i \in I}\right)'\\
      &\Leftrightarrow\quad & x &\in S \setminus \left(\bigcup (A_i)_{i \in I}\right)\\
      &\Leftrightarrow\quad & x &\in S \land \lnot \left(x \in \left(\bigcup (A_i){i \in I}\right)\right)\\
      &\Leftrightarrow\quad & x &\in S \land \lnot (\exists D \in (A_i)_{i \in I}: x \in D)\\
      &\Leftrightarrow\quad & x &\in S \land \lnot (\exists i \in I: x \in A_i)\\
      &\Leftrightarrow\quad & x &\in S \land \forall i \in I: x \not\in A_i
    \end{alignat*}
  \end{minipage}
  \hskip 0.01\textwidth
  \vline
  \hskip 0.01\textwidth
  \begin{minipage}[t]{0.49\textwidth}
    \begin{alignat*}{3}
                            && x &\in \bigcap\left(A_i'\right)_{i \in I}\\
      &\Leftrightarrow\quad & \forall D &\in (A_i')_{i \in I}: x \in D\\
      &\Leftrightarrow\quad & \forall i &\in I: x \in A_i'\\
      &\Leftrightarrow\quad & \forall i &\in I: x \in (S \setminus A_i)\\
      &\Leftrightarrow\quad & \forall i &\in I: x \in S \land x \not\in A_i\\
      &\Leftrightarrow\quad & x &\in S \land \forall i \in I: x \not\in A_i
    \end{alignat*}
  \end{minipage}
  \qed
\end{proof}

\begin{lemma}
  \[
    \left(\bigcap (A_i)_{i \in I}\right)' = \bigcup \left(A_i'\right)_{i \in I}
  \]
\end{lemma}
\begin{proof}
  As above.

  \begin{minipage}[t]{0.49\textwidth}
    \begin{alignat*}{3}
                           && x &\in \left(\bigcap (A_i)_{i \in I}\right)'\\
      &\Leftrightarrow\quad & x &\in S \setminus \left(\bigcap (A_i)_{i \in I}\right)\\
      &\Leftrightarrow\quad & x &\in S \land \lnot \left(x \in \left(\bigcap (A_i)_{i \in I}\right)\right)\\
      &\Leftrightarrow\quad & x &\in S \land \lnot (\forall D \in (A_i)_{i \in I}: x \in D)\\
      &\Leftrightarrow\quad & x &\in S \land \lnot (\forall i \in I: x \in A_i)\\
      &\Leftrightarrow\quad & x &\in S \land \exists i \in I: x \not\in A_i
    \end{alignat*}
  \end{minipage}
  \hskip 0.01\textwidth
  \vline
  \hskip 0.01\textwidth
  \begin{minipage}[t]{0.49\textwidth}
    \begin{alignat*}{3}
                           && x &\in \bigcup \left(A_i'\right)_{i \in I}\\
      &\Leftrightarrow\quad & \exists D &\in (A_i')_{i \in I}: x \in D\\
      &\Leftrightarrow\quad & \exists i &\in I: x \in A_i'\\
      &\Leftrightarrow\quad & \exists i &\in I: x \in S \setminus A_i\\
      &\Leftrightarrow\quad & \exists i &\in I: x \in S \land x \not\in A_i\\
      &\Leftrightarrow\quad & x &\in S \land \exists i \in I: x \not\in A_i\\
    \end{alignat*}
  \end{minipage}
  \qed
\end{proof}

\section*{Exercise 3}
\begin{lemma}
  $A \subseteq \mathcal{P}(\bigcup A)$
\end{lemma}
\begin{proof}
  \begin{alignat*}{3}
                         && A \subseteq \mathcal{P}(\bigcup A) \\
    &\Leftrightarrow\quad & \forall x \in A: x \in \mathcal{P}(\bigcup A)\\
    &\Leftrightarrow\quad & \forall x \in A: x \subseteq \bigcup A\\
    &\Leftrightarrow\quad & \forall x \in A: \forall y \in x: y \in \bigcup A\\
    &\Leftrightarrow\quad & \forall x \in A: \forall y \in x: \exists D \in A: y \in D
  \end{alignat*}

  Let arbitrary $x \in A$ and arbitrary $y \in x$. It suffices to show that
  $\exists D \in A: y \in D$. Indeed $D = x$.
  \qed
\end{proof}

We observe that the above lemma holds with equality when $A$ is the powerset of
some set:

\begin{lemma}
  $A = \mathcal{P}(\bigcup A) \leftrightarrow \exists B: A = \mathcal{P}(B)$
\end{lemma}
\begin{proof}
  \item $(\rightarrow)$ Letting $B = \bigcup A$ satisfies the condition.
  \item $(\leftarrow)$ Suppose $A = \mathcal{P}(B)$ for some $B$. Then
        $\bigcup A = B$ and the condition holds.
  \qed
\end{proof}

\begin{lemma}
  $\mathcal{P}(A) \cap \mathcal{P}(B) = \mathcal{P}(A \cap B)$
\end{lemma}
\begin{proof}
  \begin{minipage}[t]{0.49\textwidth}
    \begin{alignat*}{3}
                           && x \in \mathcal{P}(A) \cap \mathcal{P}(B)\\
      &\Leftrightarrow\quad & x \in \mathcal{P}(A) \land x \in \mathcal{P}(B)\\
      &\Leftrightarrow\quad & x \subseteq A \land x \subseteq B\\
      &\Leftrightarrow\quad & (\forall y \in x: y \in A) \land (\forall y \in x: y \in B)\\
      &\Leftrightarrow\quad & \forall y \in x: (y \in A \land y \in B)
    \end{alignat*}
  \end{minipage}
  \hskip 0.01\textwidth
  \vline
  \hskip 0.01\textwidth
  \begin{minipage}[t]{0.45\textwidth}
    \begin{alignat*}{3}
                           && x \in \mathcal{P}(A \cap B)\\
      &\Leftrightarrow\quad & x \subseteq A \cap B\\
      &\Leftrightarrow\quad & \forall y \in x: y \in A \cap B\\
      &\Leftrightarrow\quad & \forall y \in x: (y \in A \land y \in B)
    \end{alignat*}
  \end{minipage}
  \qed
\end{proof}

\begin{lemma}
  $\mathcal{P}(A) \cup \mathcal{P}(B) \subseteq \mathcal{P}(A \cup B)$
\end{lemma}
\begin{proof}
  \begin{minipage}[t]{0.49\textwidth}
    \begin{alignat*}{3}
                           && x \in \mathcal{P}(A) \cup \mathcal{P}(B)\\
      &\Leftrightarrow\quad & x \in \mathcal{P}(A) \lor x \in \mathcal{P}(B)\\
      &\Leftrightarrow\quad & x \subseteq A \lor x \subseteq B\\
      &\Leftrightarrow\quad & (\forall y \in x: y \in A) \lor (\forall y \in x: y \in B)
    \end{alignat*}
  \end{minipage}
  \hskip 0.01\textwidth
  \vline
  \hskip 0.01\textwidth
  \begin{minipage}[t]{0.45\textwidth}
    \begin{alignat*}{3}
                           && x \in \mathcal{P}(A \cup B)\\
      &\Leftrightarrow\quad & x \subseteq A \cup B\\
      &\Leftrightarrow\quad & \forall y \in x: y \in A \cup B\\
      &\Leftrightarrow\quad & \forall y \in x: (y \in A \lor y \in B)
    \end{alignat*}
  \end{minipage}

  ~

  We observe that $(\forall y \in x: y \in A) \lor (\forall y \in x: y \in B)
  \rightarrow \forall y \in x: (y \in A \lor y \in B)$.
  \qed
\end{proof}

In this case, equality holds if $A$ and $B$ have a subset relationship, as made
formal in the following lemma:

\begin{lemma}
  $
  (\mathcal{P}(A) \cup \mathcal{P}(B) = \mathcal{P}(A \cup B))
  \leftrightarrow
  (A \subseteq B \lor B \subseteq A)
  $
\end{lemma}
\begin{proof}
  \item $(\rightarrow)$
  Suppose for contradiction that the sets do not have a subset relationship.
  Then there are elements $x \in A \setminus B$ and $y \in B \setminus A$.
  Then $\{x, y\}$ is in $\mathcal{P}(A \cup B)$ but in neither of
  $\mathcal{P}(A), \mathcal{P}(B)$.

  \item $(\leftarrow)$
  Suppose without loss of generality that $A \subseteq B$. Then $A \cup B = B$
  and $\mathcal{P}(A \cup B) = \mathcal{P}(B)$.
  For the powersets we have that $\mathcal{P}(A) \subseteq \mathcal{P}(B)$ (by
  Lemma~\ref{lem.powerset-subset}) and
  $\mathcal{P}(A) \cup \mathcal{P}(B) = \mathcal{P}(B)$. The equality follows.

  \qed
\end{proof}

\begin{lemma}\label{lem.powerset-subset}
  $A \subseteq B \leftrightarrow \mathcal{P}(A) \subseteq \mathcal{P}(B)$
\end{lemma}
\begin{proof}
  \item $(\rightarrow)$
  Suppose $A \subseteq B$. Let $D \in \mathcal{P}(A)$. It suffices to prove that
  $D \in \mathcal{P}(B)$. We know that $D \subseteq A$. By the transitivity of
  the subset relation, we obtain that $D \subseteq B$, which proves that $D \in
  \mathcal{P}(B)$.

  \item $(\leftarrow)$
  \begin{alignat*}{3}
                         && \mathcal{P}(A) \subseteq \mathcal{P}(B)\\
    &\Leftrightarrow\quad & \forall x \in \mathcal{P}(A): x \in \mathcal{P}(B)\\
    &\Leftrightarrow\quad & \forall x: (x \subseteq A \rightarrow x \subseteq B)\\
    &\Leftrightarrow\quad & \forall x: ((\forall y \in x: y \in A) \rightarrow (\forall y \in x: y \in B))
  \end{alignat*}

  Letting $x = \{a\}$ for arbitrary $a$ in the above, we obtain that
  $(\forall y \in \{a\}: y \in A) \rightarrow (\forall y \in \{a\}: y \in B)$,
  which, by extensionality, gives
  $a \in A \rightarrow a \in B$. Since $a$ was arbitrary, we obtain that
  $\forall a: a \in A \rightarrow a \in B$, which, by definition, gives $A \subseteq B$.
  \qed
\end{proof}

\begin{lemma}
  $A = B \leftrightarrow \mathcal{P}(A) = \mathcal{P}(B)$
\end{lemma}
\begin{proof}
  \item $(\rightarrow)$ By the logical axioms of equality.
  \item $(\leftarrow)$ Let $\mathcal{P}(A) = \mathcal{P}(B)$ and suppose for
  contradiction that $A \neq B$. Then suppose without loss of generality that
  $x \in A \setminus B$.
  Then $\{x\} \in \mathcal{P}(A) \setminus \mathcal{P}(B)$.
  \qed
\end{proof}

\section*{Exercise 4}
\begin{lemma}
  $\emptyset$ is transitive.
\end{lemma}
\begin{proof}
  Vacuously. \qed
\end{proof}

\begin{lemma}\label{lem:transitive-cup}
  If $A$ is transitive, then so is $A \cup \{A\}$.
\end{lemma}
\begin{proof}
  Let arbitrary $x \in A \cup \{A\}$. If $x \in A$, then by the transitivity of
  $A$ we know that $x \subseteq A$, therefore $x \subseteq A \cup \{A\}$. If
  $x \not\in A$ then $x = A$, and so $x \subseteq A \cup \{A\}$.
  \qed
\end{proof}

\begin{lemma}
  If $A$ is transitive, then so is $\bigcup (A \cup \{A\})$.
\end{lemma}
\begin{proof}
  Let arbitrary $x \in \bigcup (A \cup \{A\})$. Then there is some
  $D \in A \cup \{A\}$ such that $x \in D$. From Lemma~\ref{lem:transitive-cup},
  $A \cup \{A\}$ is transitive, therefore $D \subseteq A \cup \{A\}$. Therefore
  $x \in A \cup \{A\}$, and so $x \subseteq \bigcup (A \cup \{A\})$.
  \qed
\end{proof}

\section*{Exercise 5}
The set theoretic predicate satisfied exactly by those functions
$f: A \longrightarrow B$ is illustrated in Formula~\ref{formula:func}. While the
formula can be written in a shorter form (observing that, for instance, one can
use the projection operators on the ordered pair for totality verification), we
feel the verbose form where the same construction is used for the same semantics
aids in readability.

\begin{figure}
\begin{algorithm}[H]
  \caption{\label{formula:func}
      The well-formed set theoretic formula $\Phi(f)$ describing a function
      $f: A \rightarrow B$}
  \begin{algorithmic}[0]
    \State{$($}
    \Indent
      \State{$\forall v_1: \forall v_2: ($}
      \Comment{Look at every element of the function}
      \Indent
        \State{$(v_1 \in f \land v_2 \in f) \rightarrow ($}
        \Indent
          \State{$\exists x_1: \exists y_1: \exists x_2: \exists y_2:$}
          \Indent
            \State{$($}
            \Indent
              \State{$x_1 \in A \land y_1 \in B$}
              \Comment{Verify domain and range}
            \EndIndent
            \State{$) \land ($}
            \Comment{Unpack $\left<x_1, y_1\right> = v_1$}
            \Indent
              \State{$\exists X_1: \exists Y_1: ($}
              \Indent
                \State{$(x_1 \in X_1 \land \forall z \in X_1: z = x_1) \land$}
                \Comment{Ensure $X_1 = \{x_1\}$}
                \State{$(x_1 \in Y_1 \land y_1 \in Y_1 \land \forall z: (z \in Y_1 \rightarrow (z = x_1 \lor z = y_1)))\land$}
                \Comment{Ensure $Y_1 = \{x_1, y_1\}$}
                \State{$(X_1 \in v_1 \land Y_1 \in v_1 \land \forall z: (z \in v_1 \rightarrow (v_1 = X_1 \lor z = Y_1)))$}
                \Comment{Ensure $v_1 = \{X_1, Y_1\}$}
              \EndIndent
              \State{$)$}
            \EndIndent
            \State{$) \land ($}
            \Comment{Unpack $\left<x_2, y_2\right> = v_2$}
            \Indent
              \State{$\exists X_2: \exists Y_2: ($}
              \Indent
                \State{$(x_2 \in X_2 \land \forall z \in X_2: z = x_2) \land$}
                \Comment{Ensure $X_2 = \{x_2\}$}
                \State{$(x_2 \in Y_2 \land y_2 \in Y_2 \land \forall z: (z \in Y_2 \rightarrow (z = x_2 \lor z = y_2)))\land$}
                \Comment{Ensure $Y_2 = \{x_2, y_2\}$}
                \State{$(X_2 \in v_2 \land Y_2 \in v_2 \land \forall z: (z \in v_2 \rightarrow (v_2 = X_2 \lor z = Y_2)))$}
                \Comment{Ensure $v_2 = \{X_2, Y_2\}$}
              \EndIndent
              \State{$)$}
            \EndIndent
            \State{$) \land ($}
            \Indent
              \State{$x_1 = x_2 \rightarrow y_1 = y_2$}
              \Comment{Ensure the function is single-valued}
            \EndIndent
            \State{$)$}
          \EndIndent
        \EndIndent
        \State{$)$}
      \EndIndent
      \State{$)$}
    \EndIndent
    \State{$) \land ($}
    \Indent
      \State{$\forall x: ($}
      \Indent
        \State{$x \in A \rightarrow ($}
        \Comment{Ensure the function is total}
        \Indent
          \State{$\exists y: ($}
            \Indent
              \State{$y \in B \land ($}
              \Indent
                \State{$\exists V: ($}
                \Indent
                  \State{$V \in f \land$}
                  \State{$\exists X: \exists Y: ($}
                  \Comment{Unpack $V = \left<x, y\right>$}
                  \Indent
                    \State{$(x \in X \land \forall z \in X: z = x) \land$}
                    \Comment{Ensure $X = \{x\}$}
                    \State{$(x \in Y \land y \in Y \land \forall z: (z \in Y \rightarrow (z = x \lor z = y)))\land$}
                    \Comment{Ensure $Y = \{x, y\}$}
                    \State{$(X \in v \land Y \in v \land \forall z: (z \in v \rightarrow (v = X \lor z = Y)))$}
                    \Comment{Ensure $v = \{X, Y\}$}
                  \EndIndent
                  \State{$)$}
                \EndIndent
                \State{$)$}
              \EndIndent
              \State{$)$}
            \EndIndent
          \State{$)$}
        \EndIndent
        \State{$)$}
      \EndIndent
      \State{$)$}
    \EndIndent
    \State{$)$}
  \end{algorithmic}
\end{algorithm}
\end{figure}

\section*{Exercise 6}
\begin{lemma}
  $\forall A: (A \neq \emptyset \rightarrow \exists \bigcap A)$.
\end{lemma}
\begin{proof}
  As $A \neq \emptyset$, therefore $\exists x \in A$, so let $x \in A$.
  By the \emph{axiom schema of specification}, we obtain the existence of
  $\{y: y \in x \land \forall z \in A: y \in z\}$. But this is $\bigcap A$.
  \qed
\end{proof}

\begin{lemma}
  $\forall A: \exists \{x \in A: \mathcal{P}(x)\}$.
\end{lemma}
\begin{proof}
  By the \emph{axiom schema of replacement} applying $\mathcal{P}$ on $A$.
  \qed
\end{proof}

\begin{lemma}
  $\forall A, B: \exists \{A \cap x: x \in B\}$
\end{lemma}
\begin{proof}
  By the \emph{axiom schema of replacement} applying $A \cap x$ on $B$.
  \qed
\end{proof}

\begin{lemma}\label{lem:cartesian-existence}
  $\forall A, B: \exists A \times B$
\end{lemma}
\begin{proof}
  By the \emph{axiom of union} we obtain the existence of $A \cup B$. By the
  \emph{axiom of power set} we obtain the existence $\mathcal{P}(A \cup B)$ and
  subsequently $\mathcal{P}(\mathcal{P}(A \cup B))$. Applying the \emph{axiom
  schema of specification}, we obtain the existence of

  \[
    \{x: x \in \mathcal{P}(\mathcal{P}(A \cup B)) \land \exists a \in A: \exists b \in B: x = \left<a, b\right>\}
    ~.
  \]

  But this is $A \times B$.
  \qed
\end{proof}

\begin{lemma}\label{lem:simple-relations-existence}
  $\forall A, B: \exists \{R: R \text{ is a relation between } A \text{ and } B\}$
\end{lemma}
\begin{proof}
  From Lemma~\ref{lem:cartesian-existence}, we obtain the existence of
  $A \times B$. Applying the \emph{axiom of power set}, we obtain the existence
  of $\mathcal{P}(A \times B)$, which is the set in question.
  \qed
\end{proof}

\begin{lemma}\label{lem:all-relations-existence}
  $\forall A, B: \exists \{X \times Y: X \in A \land Y \in B\}$
\end{lemma}
\begin{proof}
  From the \emph{axiom of union}, $\bigcup A$ and $\bigcup B$ exist. Applying
  Lemma~\ref{lem:cartesian-existence}, we deduce that
  $(\bigcup A) \times \bigcup B$ exists. From the \emph{axiom of power set},
  we obtain the existence of $\mathcal{P}((\bigcup A) \times \bigcup B)$.

  Consider some $X \in A$ and $Y \in B$. $X \times Y$ is the complete relation
  between $X$ and $Y$. But any relation between $X$ and $Y$ is also a relation
  between $\bigcup A$ and $\bigcup B$, hence in
  $\mathcal{P}((\bigcup A) \times (\bigcup B))$.
  Therefore, we can apply the \emph{axiom schema of specification} to obtain the
  existence of

  \[
  \{x: x \in \mathcal{P}((\bigcup A) \times \bigcup B) \land \exists X \in A: \exists Y \in B: x = X \times Y\}
  ~.
  \]

  This is the set in question.
  \qed
\end{proof}

\begin{lemma}
  $\forall A, B: \exists ~^A B$
\end{lemma}
\begin{proof}
  From Lemma~\ref{lem:simple-relations-existence}, we obtain the existence of
  $S = \{R: R \text{ is a relation between } A \text{ and } B\}$. But every
  function is a relation. Applying the \emph{axiom schema of specification} with
  the predicate illustrated in Formula~\ref{formula:func} requiring $f$ to be a
  function $A \longrightarrow B$, we obtain the existence of $\{x: x \in S
  \land \Phi(x)\}$, which is the set in question.
  \qed
\end{proof}

\section*{Exercise 7}
\begin{lemma}\label{lem:one-to-one-subset}
  Let $f: A \longrightarrow B$ be a function and $X \subseteq A$. Then
  $X \subseteq f^{-1}[f[X]]$.
\end{lemma}
\begin{proof}
  Let arbitrary $x \in X$. As $X \subseteq A$, so $x \in A$, therefore there
  exists some $y$ such that $\left<x, y\right> \in f$. Hence $y \in f[X]$.
  But $\left<y, x\right> \in f^{-1}$, so $x \in f^{-1}[f[X]]$.
  \qed
\end{proof}

Equality will hold for the subsets $X$ the following property, illustrated in
Figure~\ref{fig:break-range}.

\begin{figure}[H]\label{fig:break-range}
    \caption{A function breaking the equality of Lemma~\ref{lem:range-eq}.
             An element outside of $X$ has the same image as an element of $X$.}
    \centering
    \includegraphics[width=0.25\columnwidth,keepaspectratio]{break-range.pdf}
\end{figure}

\begin{lemma}\label{lem:range-eq}
  Let $f: A \longrightarrow B$ be a function and $X \subseteq A$.
  $X = f^{-1}[f[X]]$
  iff
  $\forall x \in X: \forall x' \in A \setminus X: f(x) \neq f(x')$.
\end{lemma}
\begin{proof}
  \item $(\rightarrow)$
  Suppose $X = f^{-1}[f[X]]$.
  Let arbitrary $x \in X, x' \in A \setminus X$ and suppose for contradiction
  that $f(x) = f(x')$. Then $f(x) \in f[X]$ and so $x' \in f^{-1}[f[X]]$.
  But then $x' \in X$, which is a contradiction.

  \item $(\leftarrow)$
  Suppose
  $\forall x \in X: \forall x' \in A \setminus X: f(x) \neq f(x')$.
  Let arbitrary $x \in f^{-1}[f[X]]$. Then there is some $y \in f[X]$ such that
  $\left<y, x\right> \in f^{-1}$, and so $\left<x, y\right> \in f$. But then
  there is some $x' \in X$ such that $\left<x', y\right> \in f$. Therefore $x
  \in X$. So $f^{-1}[f[X]] \subseteq X$ and therefore, applying
  Lemma~\ref{lem:one-to-one-subset}, we obtain $X = f^{-1}[f[X]]$.
  \qed
\end{proof}

\begin{lemma}
  Let $f: A \longrightarrow B$ be a function and $Y \subseteq B$. Then
  $f[f^{-1}[Y]] \subseteq Y$.
\end{lemma}
\begin{proof}
  Let arbitrary $y \in f[f^{-1}[Y]]$. Then
  $\exists x \in f^{-1}[Y]: \left<x, y\right> \in f$. As
  $x \in f^{-1}[Y]$, therefore $\exists y' \in Y: \left<y', x\right> \in
  f^{-1}$. But then $\left<x, y'\right> \in f$. By virtue of $f$ being a
  function, we have that $y' = y$ and so $y \in Y$.
  \qed
\end{proof}

Equality will hold for the subsets $Y$ with the following property, illustrated
in Figure~\ref{fig:break-domain}.

\begin{figure}[H]\label{fig:break-domain}
    \caption{A function breaking the equality of Lemma~\ref{lem:domain-eq}.
             An element of $Y$ does not have a preimage.}
    \centering
    \includegraphics[width=0.25\columnwidth,keepaspectratio]{break-domain.pdf}
\end{figure}

\begin{lemma}\label{lem:domain-eq}
  Let $f: A \longrightarrow B$ be a function and $Y \subseteq B$.
  $f[f^{-1}[Y]] = Y$
  iff
  $\restr{f}{f^{-1}[Y]}$ is \emph{onto} $Y$.
\end{lemma}
\begin{proof}
  \item $(\rightarrow)$
  Suppose $f[f^{-1}[Y]] = Y$. Let arbitrary $y \in Y$. Then
  $y \in f[f^{-1}[Y]]$. Therefore, there is some $x \in f^{-1}[Y]$ such that
  $\left<x, \right> \in f$. But this means that
  $\left<x, y\right> \in \restr{f}{f^{-1}[Y]}$.
  \item $(\leftarrow)$
  Suppose $\restr{f}{f^{-1}[Y]}$ is \emph{onto} $Y$.
  Let arbitrary $y \in Y$. Then there exists some $x$ such that
  $\left<x, y\right> \in \restr{f}{f^{-1}[Y]}$. So $\left<x, y\right> \in f$
  and $x \in f^{-1}[Y]$, so $y \in f[f^{-1}[Y]]$.
  \qed
\end{proof}

\section*{Exercise 8}

\begin{lemma}
  Let $A$ be a set of functions such that
  $\forall f \in A: \forall g \in A: (f \subseteq g \lor g \subseteq f)$.
  Then $\bigcup A$ is a function.
\end{lemma}
\begin{proof}
  Let $v \in \bigcup A$. We will show that $v$ is an ordered pair. Indeed,
  there is some $f \in A$ such that $v \in f$. By virtue of $f$ being a
  relation, $v$ is an ordered pair. Therefore $\bigcup A$ is a relation.

  Consider two ordered pairs
  $\left<x, y_1\right>, \left<x, y_2\right> \in \bigcup A$. Then there are
  $f, g \in A$ such that $\left<x, y_1\right> \in f$ and
  $\left<x, y_2\right> \in g$. Suppose without loss of generality that
  $f \subseteq g$. Then $\left<x, y_1\right> \in g$. Therefore, $y_1 = y_2$ and
  $\bigcup A$ is a function.
  \qed
\end{proof}

\begin{lemma}
  $\forall A: \forall f: (f \in A \rightarrow f \subseteq \bigcup A)$.
\end{lemma}
\begin{proof}
  Let arbitrary $A$ and $f$ such that $f \in A$. Consider an arbitrary
  $x \in f$. We will show that $x \in \bigcup A$. It suffices to show that there
  exists some $g \in A$ such that $x \in g$. But $g = f$.
  \qed
\end{proof}

\section*{Exercise 9}

Let $f: X \longrightarrow Y$ be a function. Let $I \neq \emptyset$. Let
$(A_i)_{i \in I}$ be a family of subsets of $X$ and $(B_i)_{i \in I}$ be a
family of subsets of $Y$. The following lemmas establish the commutativity of
function application and arbitrary union or intersection.

\begin{lemma}
  Let $f, A, B, I$ be as above. Then
  $f[\bigcup (A_i)_{i \in I}] = \bigcup (f[A_i])_{i \in I}$.
\end{lemma}
\begin{proof}
  \begin{alignat*}{3}
                         && y \in f[\bigcup (A_i)_{i \in I}]\\
     &\Leftrightarrow\quad & \exists x \in X: (\left<x, y\right> \in f \land x \in \bigcup (A_i)_{i \in I})\\
     &\Leftrightarrow\quad & \exists x \in X: (\left<x, y\right> \in f \land \exists i \in I: x \in A_i)\\
     &\Leftrightarrow\quad & \exists i \in I: \exists x \in X: (\left<x, y\right> \in f \land x \in A_i)\\
     &\Leftrightarrow\quad & \exists i \in I: y \in f[A_i])\\
     &\Leftrightarrow\quad & y \in \bigcup (f[A_i])_{i \in I}
  \end{alignat*}
  \qed
\end{proof}

\begin{lemma}
  Let $f, A, B, I$ be as above. Then
  $f[\bigcap (A_i)_{i \in I}] \subseteq \bigcap (f[A_i])_{i \in I}$.
\end{lemma}
\begin{proof}
  \begin{alignat*}{3}
                         && y \in f[\bigcap (A_i)_{i \in I}]\\
     &\Leftrightarrow\quad & \exists x \in X: (\left<x, y\right> \in f \land x \in \bigcap (A_i)_{i \in I})\\
     &\Leftrightarrow\quad & \exists x \in X: (\left<x, y\right> \in f \land \forall i \in I: x \in A_i)\\
     &\Rightarrow\quad & \forall i \in I: \exists x \in X: (\left<x, y\right> \in f \land x \in A_i)\\
     &\Leftrightarrow\quad & \forall i \in I: y \in f[A_i])\\
     &\Leftrightarrow\quad & y \in \bigcap (f[A_i])_{i \in I}
  \end{alignat*}
  \qed
\end{proof}

Equality will hold if every element $y$ of $Y$ has the property that either it
is not the image of some $A_i$, or one of its pre-images belongs to all $A_i$.

\begin{lemma}\label{lem:commutativity-inverse-cup}
  Let $f, A, B, I$ be as above. Then
  $f^{-1}[\bigcup (B_i)_{i \in I}] = \bigcup (f^{-1}[B_i])_{i \in I}$.
\end{lemma}
\begin{proof}
  \begin{alignat*}{3}
                         && x \in f^{-1}[\bigcup (B_i)_{i \in I}]\\
     &\Leftrightarrow\quad & \exists y \in Y: (\left<y, x\right> \in f^{-1} \land y \in \bigcap (B_i)_{i \in I})\\
     &\Leftrightarrow\quad & \exists y \in Y: (\left<y, x\right> \in f^{-1} \land \exists i \in I: y \in B_i)\\
     &\Leftrightarrow\quad & \exists i \in I: \exists y \in Y: (\left<y, x\right> \in f^{-1} \land y \in B_i)\\
     &\Leftrightarrow\quad & \forall i \in I: y \in f[B_i])\\
     &\Leftrightarrow\quad & y \in \bigcap (f[B_i])_{i \in I}
  \end{alignat*}
  \qed
\end{proof}

\begin{lemma}
  Let $f, A, B, I$ be as above. Then
  $f^{-1}[\bigcup (B_i)_{i \in I}] = \bigcup (f^{-1}[B_i])_{i \in I}$.
\end{lemma}
\begin{proof}
  The proof follows from Lemma~\ref{lem:commutativity-inverse-cup}, as the
  two statements are identical.
  \qed
\end{proof}

\section*{Exercise 10}

\begin{lemma}
  Let $f$ be an isomorphism between posets $\left<A, \preccurlyeq_A\right>,
  \left<B, \preccurlyeq_B\right>$.
  Then $f$ preserves maximal, maximum, minimal and minimum elements, as well
  as linearity.
\end{lemma}
\begin{proof}
  \item
  \textbf{Preserving extrema.}
  Suppose $a$ is a minimal element of $A$. Then
  $\nexists a' \in A: a' \preccurlyeq_A a$.
  Suppose for contradiction that $f(a)$ were not minimal. Then there would exist
  some $b' \in B$ such that $b' \preccurlyeq_B b$. But by the isomorphism
  there would exist some $a'$ with $f(a') = b'$ such that $a' \preccurlyeq_A a$,
  which is a contradiction.

  The other three cases are treated identically:
  \begin{itemize}
  \item Let $a$ be maximal. Then
  $\nexists a' \in A: a \preccurlyeq_A a'$. So
  $\nexists b' \in B: b \preccurlyeq_B b'$ and $b$ is maximal.

  \item Let $a$ be the minimum. Then
  $\forall a' \in A: a \preccurlyeq a'$. So
  $\forall b' \in B: b \preccurlyeq b'$ and $b'$ is minimum.

  \item Let $a$ be the maximum. Then
  $\forall a' \in A: a' \preccurlyeq a$. So
  $\forall b' \in B: b' \preccurlyeq b$ and $b$ is maximum.
  \end{itemize}

  \item
  \textbf{Preserving linearity.}
  Let $\preccurlyeq_A$ be linear. Suppose for contradiction that there exist
  $b_1, b_2$ in $B$ such that
  $b_1 \not\preccurlyeq_B b_2 \land b_2 \not\preccurlyeq_B b_1$.
  But then $f^{-1}(b_1) \not\preccurlyeq_A f^{-1}(b_2) \land f^{-1}(b_2)
  \not\preccurlyeq_A f^{-1}(b_1)$. Therefore $\preccurlyeq_A$ is not linear,
  which is a contradiction.
  \qed
\end{proof}

\section*{Exercise 11}
\begin{lemma}
  Let $\left<A, \preccurlyeq_A\right>, \left<B, \preccurlyeq_B\right>$ be posets
  and $f: A \longrightarrow B$ be a function such that

  \[
  \forall x \in A: \forall y \in B:
  (x \preccurlyeq_A y \leftrightarrow f(x) \preccurlyeq_B f(y))
  ~.
  \]

  Then $f$ is one-to-one.
\end{lemma}
\begin{proof}
  Suppose for contradiction that $f$ is not one-to-one. Then there are
  $x_1, x_2 \in A$ such that $x_1 \neq x_2$ and $f(x_1) = f(x_2)$. From the
  \emph{reflexivity} of $\preccurlyeq_B$, we have $f(x_1) \preccurlyeq_B
  f(x_2)$ and so $x_1 \preccurlyeq_A x_2$. But also $f(x_2) \preccurlyeq_B
  f(x_1)$ and so $x_2 \preccurlyeq_A x_1$. By the \emph{antisymmetry} of
  $\preccurlyeq_A$ we have that $x_1 = x_2$ which is a contradiction.
  \qed
\end{proof}

\end{document}
